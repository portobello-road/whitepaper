\chapter{Semantic Versioning Standard -- Version 2.0.0-rc.1}

\paragraph{
In the world of software management there exists a dread place called "dependency hell."
The bigger your system grows and the more packages you integrate into your software,
the more likely you are to find yourself, one day, in this pit of despair.
}

\paragraph{
In systems with many dependencies, releasing new package versions can quickly become a nightmare.
If the dependency specifications are too tight, you are in danger of version lock
(the inability to upgrade a package without having to release new versions of every
dependent package). If dependencies are specified too loosely, you will inevitably be
bitten by version promiscuity (assuming compatibility with more future versions than
  is reasonable). Dependency hell is where you are when version lock and/or version
promiscuity prevent you from easily and safely moving your project forward.
}

\paragraph{
As a solution to this problem, I propose a simple set of rules and requirements
that dictate how version numbers are assigned and incremented. For this system to
work, you first need to declare a public API. This may consist of documentation or
be enforced by the code itself. Regardless, it is important that this API be clear
and precise. Once you identify your public API, you communicate changes to it with
specific increments to your version number. Consider a version format of X.Y.Z (Major.Minor.Patch).
Bug fixes not affecting the API increment the patch version, backwards compatible API
additions/changes increment the minor version, and backwards incompatible API changes
increment the major version.
}

\paragraph{
I call this system "Semantic Versioning." Under this scheme, version numbers
and the way they change convey meaning about the underlying code and what has
been modified from one version to the next.
}

\section{Semantic Versioning Specification (\emph{SemVer})}

% start a numbered list
\begin{enumerate}
  \item
    Software using Semantic Versioning MUST declare a public API. This API could 
    be declared in the code itself or exist strictly in documentation. However it 
    is done, it should be precise and comprehensive.
  \item
    A normal version number MUST take the form X.Y.Z where X, Y, and Z are 
    non-negative integers. X is the major version, Y is the minor version, 
    and Z is the patch version. Each element MUST increase numerically by 
    increments of one. For instance: 1.9.0 -$>$ 1.10.0 -$>$ 1.11.0.
  \item
    When a major version number is incremented, the minor version and patch version 
    MUST be reset to zero. When a minor version number is incremented, the patch version 
    MUST be reset to zero. For instance: 1.1.3 -$>$ 2.0.0 and 2.1.7 -$>$ 2.2.0.
  \item
    Once a versioned package has been released, the contents of that version MUST NOT 
    be modified. Any modifications must be released as a new version.
  \item
    Major version zero (0.y.z) is for initial development. Anything may change 
    at any time. The public API should not be considered stable.
  \item
    Version 1.0.0 defines the public API. The way in which the version number 
    is incremented after this release is dependent on this public API and how 
    it changes.
  \item
    Patch version Z (x.y.Z $|$ x -$>$ 0) MUST be incremented if 
    only backwards compatible bug fixes are introduced. A bug fix is defined as 
    an internal change that fixes incorrect behavior.
  \item
    Minor version Y (x.Y.z $|$ x -$>$ 0) MUST be incremented if 
    new, backwards compatible functionality is introduced to the public API. It 
    MUST be incremented if any public API functionality is marked as deprecated. 
    It MAY be incremented if substantial new functionality or improvements are 
    introduced within the private code. It MAY include patch level changes. Patch 
    version MUST be reset to 0 when minor version is incremented.
  \item
    Major version X (X.y.z $|$ X -$>$ 0) MUST be incremented if 
    any backwards incompatible changes are introduced to the public API. 
    It MAY include minor and patch level changes. Patch and minor version 
    MUST be reset to 0 when major version is incremented.
  \item
    A pre-release version MAY be denoted by appending a dash and a series of 
    dot separated identifiers immediately following the patch version. Identifiers 
    MUST be comprised of only ASCII alphanumerics and dash [0-9A-Za-z-]. Pre-release 
    versions satisfy but have a lower precedence than the associated normal version. 
    Examples: 1.0.0-alpha, 1.0.0-alpha.1, 1.0.0-0.3.7, 1.0.0-x.7.z.92.
  \item
    A build version MAY be denoted by appending a plus sign and a series of dot 
    separated identifiers immediately following the patch version or pre-release 
    version. Identifiers MUST be comprised of only ASCII alphanumerics and dash 
    [0-9A-Za-z-]. Build versions satisfy and have a higher precedence than the 
    associated normal version. Examples: 1.0.0+build.1, 1.3.7+build.11.e0f985a.
  \item
    Precedence MUST be calculated by separating the version into major, minor, 
    patch, pre-release, and build identifiers in that order. Major, minor, and 
    patch versions are always compared numerically. Pre-release and build version 
    precedence MUST be determined by comparing each dot separated identifier as 
    follows: identifiers consisting of only digits are compared numerically and 
    identifiers with letters or dashes are compared lexically in ASCII sort order. 
    Numeric identifiers always have lower precedence than non-numeric identifiers. 
    Example: 1.0.0-alpha $<$ 1.0.0-alpha.1 $<$ 1.0.0-beta.2 $<$ 1.0.0-beta.11 
    $<$ 1.0.0-rc.1 $<$ 1.0.0-rc.1+build.1 $<$ 1.0.0 $<$ 1.0.0+0.3.7 
    $<$ 1.3.7+build $<$ 1.3.7+build.2.b8f12d7 $<$ 1.3.7+build.11.e0f985a.
\end{enumerate}

\section{Why Use Semantic Versioning?}

\paragraph{
This is not a new or revolutionary idea. In fact, you probably do something close 
to this already. The problem is that "close" isn't good enough. Without compliance 
to some sort of formal specification, version numbers are essentially useless for 
dependency management. By giving a name and clear definition to the above ideas, 
it becomes easy to communicate your intentions to the users of your software. Once 
these intentions are clear, flexible (but not too flexible) dependency specifications 
can finally be made.
}

\paragraph{
A simple example will demonstrate how Semantic Versioning can make dependency hell 
a thing of the past. Consider a library called "Firetruck." It requires a Semantically 
Versioned package named "Ladder." At the time that Firetruck is created, Ladder is 
at version 3.1.0. Since Firetruck uses some functionality that was first introduced 
in 3.1.0, you can safely specify the Ladder dependency as greater than or equal to 
3.1.0 but less than 4.0.0. Now, when Ladder version 3.1.1 and 3.2.0 become available, 
you can release them to your package management system and know that they will be 
compatible with existing dependent software.
}

\paragraph{
As a responsible developer you will, of course, want to verify that any package 
upgrades function as advertised. The real world is a messy place; there's nothing 
we can do about that but be vigilant. What you can do is let Semantic Versioning 
provide you with a sane way to release and upgrade packages without having to 
roll new versions of dependent packages, saving you time and hassle.
}

\paragraph{
If all of this sounds desirable, all you need to do to start using Semantic 
Versioning is to declare that you are doing so and then follow the rules. 
Link to this website from your README so others know the rules and can benefit from them.
}

\section{FAQ}

\subsection{ How should I deal with revisions in the 0.y.z initial development phase? }

\paragraph{
The simplest thing to do is start your initial development release at 0.1.0 and then 
increment the minor version for each subsequent release.
}

\subsection{ How do I know when to release 1.0.0? }

\paragraph{
If your software is being used in production, it should probably already be 1.0.0. 
If you have a stable API on which users have come to depend, you should be 1.0.0. 
If you're worrying a lot about backwards compatibility, you should probably already be 1.0.0.
}

\subsection{ Doesn't this discourage rapid development and fast iteration? }

\paragraph{
Major version zero is all about rapid development. If you're changing the API 
every day you should either still be in version 0.x.x or on a separate development 
branch working on the next major version.
}

\subsection{ If even the tiniest backwards incompatible 
changes to the public API require a major version bump, won't I end 
up at version 42.0.0 very rapidly?
}

\paragraph{ 
This is a question of responsible development and foresight. Incompatible changes 
should not be introduced lightly to software that has a lot of dependent code. The 
cost that must be incurred to upgrade can be significant. Having to bump major 
versions to release incompatible changes means you'll think through the impact 
of your changes, and evaluate the cost/benefit ratio involved.
}

\subsection{ Documenting the entire public API is too much work! }

\paragraph{
It is your responsibility as a professional developer to properly document software 
that is intended for use by others. Managing software complexity is a hugely important 
part of keeping a project efficient, and that's hard to do if nobody knows how to 
use your software, or what methods are safe to call. In the long run, Semantic 
Versioning, and the insistence on a well defined public API can keep everyone 
and everything running smoothly.
}

\subsection{ What do I do if I accidentally release a backwards incompatible change as a minor version? }

\paragraph{
As soon as you realize that you've broken the Semantic Versioning spec, fix the 
problem and release a new minor version that corrects the problem and restores 
backwards compatibility. Remember, it is unacceptable to modify versioned releases, 
even under this circumstance. If it's appropriate, document the offending version 
and inform your users of the problem so that they are aware of the offending version.
}

\subsection{ What should I do if I update my own dependencies without changing the public API? }

\paragraph{
That would be considered compatible since it does not affect the public API. Software
that explicitly depends on the same dependencies as your package should have their 
own dependency specifications and the author will notice any conflicts. Determining 
whether the change is a patch level or minor level modification depends on whether you 
updated your dependencies in order to fix a bug or introduce new functionality. I 
would usually expect additional code for the latter instance, in which case it's 
obviously a minor level increment.
}

\subsection{ What should I do if the bug that is being fixed returns the code to 
being compliant with the public API (i.e. the code was incorrectly out of sync with 
the public API documentation)? }

\paragraph{
Use your best judgment. If you have a huge audience that will be drastically impacted 
by changing the behavior back to what the public API intended, then it may be best to 
perform a major version release, even though the fix could strictly be considered a 
patch release. Remember, Semantic Versioning is all about conveying meaning by how 
the version number changes. If these changes are important to your users, use the 
version number to inform them.
}

\subsection{ How should I handle deprecating functionality? }

\paragraph{
Deprecating existing functionality is a normal part of software development and 
is often required to make forward progress. When you deprecate part of your public 
API, you should do two things:
}
\begin{enumerate}
\item
update your documentation to let users know about the change
\item
issue a new minor release with the deprecation in place. Before you completely remove the functionality in a new major release there should be at least one minor release that contains the deprecation so that users can smoothly transition to the new API.
\end{enumerate}

\section{About}

\paragraph{
The Semantic Versioning specification is authored by Tom Preston-Werner, inventor of Gravatars and cofounder of GitHub.
}
\paragraph{
If you'd like to leave feedback, please open an issue on GitHub.
}

\section{License}

\paragraph{
Creative Commons - CC BY 3.0 http://creativecommons.org/licenses/by/3.0/
}
