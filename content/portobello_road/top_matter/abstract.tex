\begin{abstract}
The Portobello Road protocol, simply put, enables truly free markets.

Portobello Road is built on a multihub-client model. Hubs are individual servers which
may be clustered together or connected into federations of optionally clustered hubs.
Clients connect to one or more of these federations and coalesce their available information, initiate
transactions and coordinate communications amongst identities in the system.

Communications between identities are protected by a modified OTR messaging protocol which
provides for deniable secure communications by which to arrange transactions.
Because of this, network and network service providers, such as escrow agents, remain ignorant
about the true nature of the transactions undertaken by their clients. The only parties with
access to the details of the transaction are buyer and merchant and there is no way for the
merchant to prove to others after the fact that the buyer did indeed create the specified transaction.
Escrow decisions are put directly into the hands of the counterparties involved by requiring the creation of
a Nash equilibrium between them (also known as Mutually Assured Destruction). This allows escrow to be automated
and escrow agents are no longer required to be adjudicators, only to follow the escrow protocol and
to not abscond with the money in transit. If m-of-n signatures are required for the
settlement method, escrow agents only need to trusted to follow the escrow protocol.

Distribution is acheived via federation and clustering. Federation is a special case of
clustering in which hub operators may put in place policies which affect the traffic flowing
between clusters of hubs. This permits hub operators the freedom to charge for access to their
network of merchants and for carrying transaction information between identities. This is
effectively how Internet peering works today. Addtionally, hub operators may charge clients for access to
their network of merchants and may, as they are no longer needed to provide the escrow
functionality centralised marketplaces typically provide.

By these mechanisms, Portobello Road provides for a need-to-know market in which buyers and
sellers may transact, escrow and settle freely and without coercion. Peer-to-peer review and attestation provide
for reputation and decentralised escrow protects against malicious entities while allowing individuals
to settle disputes in the manor they choose. Users are not coerced into staying with one organisation or
service provider but are free to chose the best providers where best is defined by them and no one else.
\end{abstract}
