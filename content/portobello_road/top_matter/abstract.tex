\begin{abstract}
The Portobello Road protocol, also known as Portobello Road, enables truly free
markets.

Portobello Road is built on a multihub-client model. Hubs are individual servers
which may be clustered together or connected into federations of optionally
clustered hubs. Clients connect to one or more of these federations and coalesce
their available information, initiate transactions and coordinate communications
amongst entities in the system.

Communications between entities are protected by a modified OTR messaging
protocol which provides for deniable secure communications by which to arrange
transactions asynchronously. Because of this, network and network service
providers, such as escrow agents, remain ignorant about the true nature of the
transactions undertaken by their clients. The only parties with access to the
details of the transaction are those entities directly envolved in the transaction
of physical goods and services and there is no way for a single participant in
the transaction to prove to others after the fact that the other party did 
indeed create the specified transaction.

Escrow decisions are put directly into the hands of the entities involved
by requiring the creation of a Nash equilibrium between them (a technique
also known as Mutually Assured Destruction). This allows escrow to be automated
and escrow agents are no longer required to be adjudicators in the case of
disagreement, only to follow the escrow protocol and to not abscond with the
money in transit. If multiparty authorisation is supported by the agreed upon
settlement method, escrow agents need only be trusted to follow the escrow
protocol -- adherance to which is verifiable. The value wagered is not necessarily
equal to the value exchanged by parties to the transaction.

Unlike traditional escrow, in this system escrow agents do not remit to a party
other than from which the wagered funds were received. Should a transaction be
successful, the funds which created the Nash equilibrium between transacting
parties are returned to the sending party. Should a transaction
fail, the funds are `destroyed' according to each agent's policy at the direction
of either transaction participant -- a policy agreed to by all parties. An
escrow agent may either choose to keep `destroyed' funds or not,
with the ascent of all parties. Finally, an escrow agent may choose to require
direct payment from both, either or neither other party to the transaction.
Market forces are expected to decide which mechanism for incentivising escrow
agents is more appropriate.

Distribution is acheived via federation. Federation connects hubs
in a way in which hub operators may put in place policies which affect and measure
the messages flowing between hubs. This permits hub operators the
freedom to charge or to not charge for access to their network of merchants and
for carrying transaction information between identities. This is effectively how
Internet peering works today. Addtionally, hub operators may charge clients for
access to their network of merchants and may, as they are no longer needed to
provide the escrow functionality centralised marketplaces typically provide,
charge merchants for access to their clients. This allows costs to users to
reflect the true cost of maintaining the hub system and federation model.

By these mechanisms, Portobello Road provides for a need-to-know market in which
buyers and sellers may transact, escrow and settle freely and without coercion.
Peer-to-peer review and attestation provide for reputation and decentralised
escrow protects against malicious entities while allowing individuals
to settle disputes in the manor they choose. Users are not coerced into staying
with one organisation or service provider but are free to choose and take their
data and business to the best providers where best is defined by them and no one
else.
\end{abstract}
